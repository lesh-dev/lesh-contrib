\documentclass[a4paper,12pt]{article}
\usepackage{cmap}
\usepackage{indentfirst}
\usepackage{fancyhdr}
\usepackage[T1,T2A]{fontenc}
\usepackage[utf8]{inputenc}
\usepackage[russian]{babel}
\usepackage{amsmath}
\usepackage[left=2cm,right=2cm,top=3cm,bottom=2cm]{geometry}
\usepackage[dvips]{graphicx}
\usepackage{hyperref}
\usepackage{bookmark}
%\graphicspath{{\noiseimages}}
%\usepackage{graphicx}

\headheight16pt

\pagestyle{fancy}
\fancyhead{}
\fancyhead[LO]{Межрегиональная физическая олимпиада 2014---2015. Заочный тур}
\fancyhead[RO]{Стр.~\thepage~из 2}
\fancyfoot{}

\newcommand\hr[1]{\left({#1}\right)}
\newcommand\un[1]{\,\emph{#1}}

\def\thesection{\arabic{section}.}
\def\thesubsection{\arabic{section}.\arabic{subsection}.}

% Даты исправлять только здесь
\newcommand{\olympdatestart}{1~ноября}
\newcommand{\olympyearstart}{2015}

\newcommand{\olympdateend}{1~декабря}
\newcommand{\olympyearend}{2016}

\newcommand{\olympcheckend}{20~января}

\newcommand{\olympmail}{fizleshcontest@yandex.ru}
%\newcommand{\olympmail}{contest@fizlesh.ru}


\begin{document}

\section{Теоретические задачи}

\subsection{Гром и молния}
Однажды знаменитый маг и~волшебник Дмитрий К., разгуливая по лесу, наблюдал вспышку молнии,
но не услышал грома. Есть ли в~этом магия, или такое может быть в~повседневной жизни?

\subsection{Алхимия}
Алхимик Всеволод~XV просыпал порошок меди и~стеклянный песок на пол, а~потом ещё
и~наступил на этот бардак. В~результате этот алхимический мусор был перемешан,
и~Всеволод~XV задумался, как быстро отделить одно от другого, не используя магию.
А~как бы это сделал ты?

\subsection{Пара экспериментов}
Начинающий разработчик философского камня Егор опустил в~стакан с~жидкостью кусочек
льда с~полостью внутри и~замерил уровень жидкости. Потом Егор магически заполнил
полость холодным свинцом и~снова замерил уровень жидкости. При каких условиях
внешней среды (особенности жидкости, атмосферы, эфирного ветра), постоянных
для обоих тестов, значения будут отличаться?

\subsection{Какой идти дорогой?}
Миша в~первый раз приехал на Летнюю экологическую школу преподавателем физики.
От лагеря физиков до школьной столовой есть два маршрута: один всё время идёт в~гору
под одним и~тем же углом; другой на половине пути ровный, а~вторую половину
дистанции занимает подъём. По карте маршруты имеют одинаковую длину.
По какому пути Мише лучше идти?

\subsection{Качели}
Больше всего на свете Маша любит физику и~качели! Как-то раз, качаясь на качелях,
Маша заметила, что качели скрипят только в~одном направлении на высокой ноте,
а~обратно на низкой. Немного поразмыслив, девочка нашла объяснение этому факту.
Как ты думаешь, какое?

\section{Экспериментальные задачи}

\subsection{Цена слова}
Как известно, то, сколько марок клеить на почтовый конверт, зависит не только от того,
куда письмо необходимо отправить, но и~от~веса самого письма. А~ведь каждое слово имеет
свой вес! Попробуй экспериментально определить, сколько весит написанное простым
карандашом на бумаге слово <<олимпиада>>. Для определённости будем считать, что
оно написано поперёк листа A4 крупными буквами (19~\emph{см} в~длину и~4~\emph{см} в~высоту).

\subsection{Плотность раствора}
Начинающий естествоиспытатель Лямбда решил выяснить, как зависит плотность
водного раствора какого-либо вещества, от концентрации этого вещества
в~растворе (например, соли, уж её-то везде легко достать). Присоединись
к~исследованию Лямбды и~попробуй самостоятельно собрать необходимый
для этого прибор и~провести соответствующие измерения.

\subsection{Волчок}
Если раскручивать разноцветный «волчок», то при некоторой скорости его цвета будут сливаться.
Попробуй экспериментально определить, при какой скорости это произойдёт, если <<волчок>>:\\
\vbox to 20mm{
\vss
\begin{enumerate}
\setlength{\itemsep}{-2pt}
\item[а)] двухцветный
\item[б)] восьмицветный
\item[в)] двадцитичетырёхцветный
\end{enumerate}
\vss}

\subsection{Конденсаторы}
Для построения новой модели летающей тарелки межгалактического типа гуманоиду Дане
совершенно необходимо много конденсаторов! Но вот беда: на его родной планете их нет.
Помоги гуманоиду Дане сделать из подручных материалов конденсатор ёмкостью~100~\emph{pF}.
Не забудь, что после этого необходимо (в  целях проверки) измерить эту ёмкость,
обосновать способ измерения и~оценить его точность.

\subsection{Магниты}
Глеб очень любит магниты! Он собирает любые магниты, какие только может обнаружить!
А~как их обнаружить, если не по создаваемому ими магнитному полю? Постарайся придумать
и~проверить на практике любой способ обнаружения магнитного поля
на максимальном расстоянии от его источника.


\section*{Краткие правила оформления работ}
\small
Титульный лист:
\begin{enumerate}
\setlength{\itemsep}{-3pt}
\item Фамилия, имя, отчество участника олимпиады (полностью, печатными буквами)
\item Фамилии, имена, отчества родителей (полностью)
\item Школа, класс
\item Домашний адрес полностью, с индексом, названием населённого пункта и региона
\item Контактный телефон
\item Действующий адрес электронной почты (крайне желателен для оповещения о приглашении на второй тур)
\item Название детского объединения (кружок, клуб) по физике, которое посещаете,\\
Ф.И.О. руководителей (полностью, печатными буквами)
\item Фамилия, имя, отчество учителя физики (полностью, печатными буквами)
\end{enumerate}


Решать все задачи не обязательно. Лучше максимально полно ответить на вопросы задач,
рассмотреть интересные случаи. В некоторых задачах возможно несколько решений, базирующихся
на разных идеях. За неверные версии оценка не снижается.

Возможно использование литературы и~других источников информации (в том числе сети Интернет).
Однако работа выполняется индивидуально, пользоваться помощью сверстников и учителей не разрешается.
Обращаем Ваше внимание, что в случае обнаружения признаков списывания друг у друга,
иных форм <<коллективного творчества>> и~других нарушений Правил проведения Олимпиады,
Оргкомитет оставляет за собой право дисквалифицировать участников, прошедших по сумме баллов во второй тур.

Большая просьба: \textbf{пишите разборчиво, крупно и ярко выделяйте номера задач}.
При оформлении экспериментальных задач крайне желательно предоставить
\begin{itemize}
\setlength{\itemsep}{-3pt}
\item рисунок или фотографию установки
\item схему эксперимента (последовательность действий)
\item результаты эксперимента (лучше в виде таблицы)
\end{itemize}

Каким бы способом Вы ни оформляли работу, лучше всего отослать её на проверку по~электронной
почте или через сайт (\href{http://fizlesh.ru/contest/send}{fizlesh.ru/contest/send}.
Если Вы оформляете её на бумаге, отсканируйте или сфотографируйте работу
(пожалуйста, для обеспечения читаемости, не пользуйтесь для этого камерами на телефонах).
\textbf{Важно}: в любом случае проверьте электронный вариант вашей работы на читаемость!
Решение, которое мы не сможем по той или иной причине разобрать, будет приравнено к его отсутствию.

Даже если у Вас нет возможности перевести работу в электронный вид, отправьте нам уведомление
о~том, что выслали её почтой. Свои работы или уведомления об отправке Вы можете высылать
на электронный адрес \href{mailto:contest@fizlesh.ru}{contest@fizlesh.ru}
или через форму на сайте \href{http://fizlesh.ru/contest}{fizlesh.ru/contest}.
Отправить работу почтой можно на адрес \emph{121357, г.~Москва, ул.~Кременчугская, д.~13, ГБОУ школа-интернат
<<Интеллектуал>>, Шувалову В.\,Ю.}

\end{document}
