\documentclass[12pt]{article}
\usepackage[utf8]{inputenc}
\usepackage[english,russian]{babel}
\usepackage{konduit01-utf8}

\textheight=210mm

\newcommand{\pstar}[1][]{\refstepcounter{pnum}{\immediate\write\tempfile{\arabic{znum}.\arabic{pnum}.\pstyle.\if\relax\detokenize{#1}\relax\zprev\else#1\fi}} {\bf\thepnum}$^{\mbox{#1}*}${\bf{)}$\;$}}

\begin{document}

\name[Линейные пространства]{1}{}{}

{\defin {\it Линейным простраством} (или {\it векторным пространством}) над множеством чисел $\mathbb{F}$ (обычно под числами будут подразумеваться действительные числа $\mathbb{R}$) называется множество $L$ с двумя операциями --- сложением (паре $a, b$ элементов $L$ ставится в соответствие элемент $L$, обозначаемый $a+b$) и умножением (паре $\lambda$ из F, $a$ из L ставится в соответствии элемент $L$, обозначаемый $\lambda a$) --- удовлетворяющими следующим условиям (аксиомам):
\begin{enumerate}
	\item $a+b = b+a$;
	\item $(a+b)+c = a+(b+c)$;
	\item существует такой элемент $0\in L$, что $a+0 = a$;
	\item $\forall a \  \exists b \  a+b = 0$;
	\item $\lambda(\mu a) = (\lambda\mu) a$;
	\item $\lambda (a+b) = \lambda a + \lambda b$;
	\item $1\cdot a = a$.
\end{enumerate}
Элементы линейного пространства называют {\it векторами}. Линейное пространство, состоящее из одного элемента, обозначается $0$.
}

\z Являются ли линейными пространствами
	\p многочлены с действительными коэффициентами? А многочлены степени $\le n$? А степени $>n$?
	\p Многочлены от $x$, равные в точке $x=7$ нулю? Единице? А многочлены, делящиеся на $x^2+3$?
	\p Бесконечные последовательности; ограниченные последовательности; неограниченные последовательности?
	\p Арифметические прогрессии? Геометрические прогрессии?
	\p Последовательности Фибоначчи (последовательности, удовлетворяющие условию $x_{n+2} = x_{n+1} + x_n$)?
	\p Ограниченные функции на отрезке $[0;1]$?
	\pstar $\mathbb C$ над $\mathbb R$, $\mathbb R$ над $\mathbb Q$, $\mathbb R$ над $\mathbb C$?
	






{

\defin {\it Линейным подпространством} линейного пространства $L$
называется непустое подмножество $L_1\subset L$, удовлетворяющее условиям:
\begin{enumerate}
	\item $\forall x,y\in L_1 \ x+y\in L_1$;
	\item $\forall \lambda \in F \  \forall x\in L_1 \  \lambda x\in L_1$
\end{enumerate}
}

\z Доказать, что линейное подпространство является линейным пространством (относительно тех же операций сложения и умножения на число).





{
\defin {\it Суммой} линейных подпространств $L_1$ и $L_2$ линейного пространства $L$ называется множество, обозначаемое $L_1 + L_2$ и состоящее из всех $x\in L$, представимых в виде $x = y+z$, где $y\in L_1$, $z\in L_2$.
}


\z Пусть $L_1$, $L_2$ --- линейные подпространства. Являются ли линейными подпространствами следующие множества?
	\p $L_1 + L_2$;
	\p $L_1\cup L_2$;
	\p $L_1\cap L_2$?


\z Пусть $L_1$, $L_2$, $L_3$ --- линейные подпространства. Докажите, что
	\p $L_1 + 0 = L_1 = L_1 + L_1$;
	\p $L_1\cap L_3$ + $L_2\cap L_3 \subset (L_1+L_2)\cap L_3$;
	\p приведите пример ситуации, когда два пространства из предыдущего пункта не совпадают;
	\p $L_1 + (L_2 + L_3) = (L_1 + L_2) + L_3$?

\z Найти суммы и пересечения:
	\p пространства четных и пространства нечетных функций на $\mathbb R$;
	\p пространства функций на $\mathbb R$, равных нулю на множествах $M_1$, $M_2$;
	\pstar пространства многочленов, делящихся на фиксированные многочлены $p_1,p_2\in\mathbb R[x]$.


\end{document}






